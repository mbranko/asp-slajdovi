\documentclass[compress,aspectratio=169]{beamer}
\usepackage{irbookslide}
\usepackage{irilmenau2}
\usepackage{tikz}
\usepackage{url}
\usepackage{ifxetex}
%\RequireXeTeX
\usepackage{fontspec} % zahteva paket euenc
\usepackage{xunicode}
\usepackage{xltxtra}
\usepackage{polyglossia}
%\setdefaultlanguage[script=Latin]{serbian}

\title{Algoritmi i strukture podataka}
\subtitle{Slajdovi sa predavanja}
\author{Branko Milosavljević}
\institute{Katedra za informatiku, Fakultet tehničkih nauka, Univerzitet u
Novom Sadu}
\date{2022.}
\subject{Predavanja sa ASP}

\begin{document}

\frame{\titlepage}

\section{Predmet}
\frame{
  \frametitle{Obrazovni cilj}
  \begin{itemize}
    \item poznavanje principa rada i implementiranje algoritama i struktura
    podataka u operativnoj memoriji
  \end{itemize}
}
\frame{
  \frametitle{Sadržaj predmeta}
  \begin{itemize}
    \item analiza algoritama
    \item rekurzija
    \item nizovi
    \item stekovi, redovi, dekovi
    \item liste
    \item stabla
    \item redovi sa prioritetom
    \item mape, heš tabele, skip liste
    \item stabla pretrage
    \item sortiranje
    \item obrada teksta
    \item rukovanje grafovima
    \item upravljanje memorijom i B-stabla
  \end{itemize}
}
\frame{
  \frametitle{Organizacija nastave}
  \begin{itemize}
    \item predavanja: 3 časa nedeljno (svi zajedno)
    \item vežbe: 2 časa nedeljno (po grupama)
  \end{itemize}
}
\frame{
  \frametitle{Nastavnici}
  \begin{itemize}
    \item Branko Milosavljević \\ \texttt{mbranko@uns.ac.rs}
    \item Tamara Kovačević \\ \texttt{tamara.kovacevic@uns.ac.rs} \\ \ \\
    \item konsultacije putem emaila
    \item ili uživo u unapred dogovorenom terminu
    \item ili na Microsoft Teams  % \url{http://enastava.ftninformatika.com}
  \end{itemize}
}
\frame{
  \frametitle{Literatura}
  \begin{itemize}
    \item Michael T. Goodrich, Roberto Tamassia, Michael H. Goldwasser. {\em
    Data Structures and Algorithms in Python}. Wiley, 2013.
  \end{itemize}
  \begin{center}
    \includegraphics[width=4cm]{asp-00-pic01}
  \end{center}
}
\frame{
  \frametitle{Nastavni materijal}
  \begin{itemize}
    \item sajt predmeta:
      \url{http://enastava.ftninformatika.com} \\ \ \\
    \item slajdovi sa predavanja
    \item zadaci sa vežbi
    \item domaći zadaci
  \end{itemize}
}
\frame{
  \frametitle{Polaganje ispita}
  \begin{itemize}
    \item domaći zadaci
    \item projekat \#1: sredinom semestra
    \item projekat \#2: na kraju semestra
    \item usmeni ispit
  \end{itemize}
}
\end{document}
